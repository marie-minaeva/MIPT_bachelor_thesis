\documentclass[../manuscript.tex]{subfiles}

\section{Материалы и методы}
\subsection{Cырые данные и анализ дифференциальной экспрессии}
\par В качестве входных данных инструмента используется нормализованная  матрица количества прочтений, полученная из ресурса ARCHS 4 \cite{lachmann2018} для каждого рассматриваемого перехода. С помощью широко используемого инструмента edgeR \cite{edgeR} был выполнен анализ дифференциальной экспрессии и получены данные анализа дифференциальной экспрессии.
\subsection{Построение генных сетей и подсчет метрик центральности}
\subsubsection{Получение данных о взаимодействии}
\par Для получения информации о взаимодействии генов внутри сигнатуры проводятся API запросы в базы данных STRING \cite{damian2018} и BioGRID \cite{biogrid}. API запросы производились с помощью пакета requests языка Python. В этих базах данных содержится информация о белок-белковых взаимодействиях, как физических и регуляторных (воздействия транскрипционных факторов на гены), так и предсказанных взаимодействиях. Помимо информации о взаимодействии, в базе данных STRING \cite{damian2018} содержатся так называемые коэффициенты достоверности, показывающие насколько вероятно взаимодействие между белками или генами.

\subsubsection{Построение генных сетей}
\par С помощью эффективного и действенного пакета под названием Graph-Tool \cite{graph-tool} языка Python, строятся генные сети, а также вычисляются метрики центральности. Более конкретно, метрики центральности, такие как pagerank centra\-lity, betweenness centrality, eigenvector centrality, closeness centrality, Katz centrali\-ty, Hits centrality, eigentrust, вычисляются с помощью соответствующих функций пакета на основании реконструированного неориентированного графа белок-белковых и ген-генных взаимодействий. 

\subsection{Нормализация}
\par Стандартная нормализация, использовавшаяся для уменьшения мат. ожидания и дисперсии $\log{\text{FC}}$ была взята из пакета scikit-learn \cite{scikit-learn} языка Python. Была использована функция StandardScaler, которая центрирует и нормализуют входную матрицу (делает мат.ожидание равным нулю и стандартное отклонение равным единице).

\subsection{Статистический анализ}
\par Для оценки статистической значимости получаемых результатов был использован тест Колмогорова-Смирнова, реализованный средствами пакета scipy языка Python в виде функции ks\_2samp. Данная функция вычисляет статистику Колмогорова-Смирнова для двух выборок.
Это двусторонняя проверка нулевой гипотезы о том, что две независимые выборки взяты из одного и того же непрерывного распределения.

\subsection{Байесовская оптимизация}
\par Байесовская оптимизация проводилась с помощью пакета skopt языка Python функцией gp\_minimize. Так как в ходе реализации инструмента стояла задача максимизации статистики Колмогорова-Смирнова, а функция gp\_minimize предназначена для минимизации оптимизируемой функции, то для оптимизации использовалась следующая функция: - статистика Колмогорова-Смирнова.

\subsection{Оценка качества модели}
\par Метрика GSEA enrichment score \cite{gsea} использовалась для оценки качества ранжирования модели. Она показывает насколько представлен выбранный список в ранжированном списке. Метрика GSEA рассчитывается следующим образом:
\begin{itemize}
    \item Упорядочивается N генов в D, чтобы сформировать $L=\{g_1, ..., g_{N}\}$ в соответствии с корреляцией $r (g_j) = r_j$ их профилей экспрессии с C.
    \item Оценивается доля генов в S («совпадениях»), взвешенная по их корреляция и доля генов, не входящих в S («промахи»), присутствующих до данной позиции i в L.
$$P_{hit}(S, i) = \sum_{\substack{g_j \in S\\ j \leq i}} \frac{|r_j|^p}{N_R}, \,\,\,\,\,\,\,\,\,\,\,\text{where}\,\,\,\,N_R = \sum_{g_j \in S} |r_j|^p$$

$$P_{miss}(S, i) = \sum_{\substack{g_j \notin S\\ j \leq i}} \frac{1}{(N - N_H)}$$
    ES - это максимальное отклонение $P_{hit} - P_{miss}$ от нуля. Для
    случайно распределенная S, ES (S) будет относительно небольшой, но если она сосредоточены в верхней или нижней части списка или иным образом распределены неслучайно, тогда ES (S) будет соответственно высоким. Когда $p = 0$, ES (S) сводится к стандартной статистике Колмогорова – Смирнова; когда $p = 1$, гены в S взвешены по их корреляции с C, нормированным на сумму корреляций по всем генам из S.
\end{itemize}
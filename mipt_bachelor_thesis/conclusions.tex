\documentclass[../manuscript.tex]{subfiles}
\section{Заключение. План дальнейших исследований}

\par В ходе работы была реализована модель инструмента, основанного на подходе CMap, а также анализе топологических метрик центральности сетей белок-белковых и ген-генных взаимодействий.

\par В процессе создания была проведена широкомасштабная оптимизация коэффициентов в выражении для коэффициентов влиятельности. После нахождения оптимальных коэффициентов для каждого клеточного перехода коэффициенты в выражении для inf\_score были усреденены.

\par После оптимизации была произведена валидация полученного инструмента с помощью нашего ''золотого стандарта'', извлеченного из базы данных CFM, как для оптимальных коэффициентов для каждого перехода, так и для усредненных значений. 

\par Помимо валидации, также была проверена применимость TopoCMap для поиска соединений исходя из их механизма действия на клетки. Было установлено, что инструмент весьма успешно приоритизирует соединения исходя из механизма действия на уровне транскриптома. Однако не все механизмы действия успешно приоритизируются нашим инструментом. Это может быть связано с нестабильностью транскриптомных сигнатур при воздействии одним и тем же химическим агентом на различные клеточные линии, при различном времени инкубирования клеточной линии в химическом агенте, а также вариабельностью транскриптомных данных от эксперимента к эксперименту.

\par Вдобавок, было проведено сравнение нашего инструмента с инструментом, предложенным в статье Lamb et al. \cite{lamb2006}. Несмотря на ряд сложностей при сравнении инструментов, вызванных различиями баз данных сигнатур этих инструментов, было показано, инструмент TopoCMap показывает результаты не хуже тех, что представлены в статье. В некоторых случаях, наш инструмент показывает более высокие результаты за счет того, что вероятность случайно ранжировать вещества верным образом для нашего инструмента на порядок ниже, чем для инструмента Lamb et al. Такое понижение вероятности случайности ранжирования вызвано существенным увеличением базы данных сигнатур нашего инструмента по сравнению с инструментом Lamb et al.

\par В дальнейшем планируется улучшить наш инструмент. Также планируется добиться статистической значимости предсказаний при валидации на клеточных переходах, описанных выше. Кроме того, планируется объединение TopoCMap с инструментом предсказания синергического эффекта на основе подхода CMap, также разрабатываемого в нашей лаборатории.
\documentclass[../manuscript.tex]{subfiles}

\section{Обозначения, сокращения, основные определения}
WLN - Wiswesser\newline
SLN - Sybyl\newline
HBLC - high betweenness and low connectivity\newline
CMap - Connectivity Map\newline
HDAC - Деацетилазы гистонов\newline
HSP90 - белок теплового шока 90\newline
MCF7 - клеточная линия рака груди\newline
CD - Characterictical Direction\newline
CDK1-2 - циклин-зависимые киназы 1-2\newline
GSK3$\beta$ - Киназа гликоген синтазы 3 $\beta$\newline
STAT - signal transducer and activator of transcription protein\newline
GSEA - gene set enrichment analysis\newline
ES - enrichment score\newline
pr - pagerank centrality\newline
bw - betweenness\newline
ev - eigenvector centrality\newline
cl - closeness\newline
kz - Katz centrality\newline
hs - Hits centrality\newline
et - eigentrust\newline
FC - кратное изменение\newline
inf\_score - коэффициент влиятельности\newline
fb - фибробласты\newline
heart - индуцированные кардиомиоциты\newline
neuron - индуцированные нейроны\newline
neural - индуцированные нейральные стволовые клетки\newline
beta - индуцированные панкреатические бета клетки\newline
ips - индуцированные плюрипотентные стволовые клетки\newline
mes - мезенхимальные стволовые клетки\newline
TGF-$\beta$ - трансформирующий фактор роста бета\newline
NF$\kappa$B - ядерный фактор «каппа-би»\newline
MAPK - митоген-активируемая протеинкиназа\newline
PI3K - Фосфоинозитид-3-киназы\newline
Akt - протеинкиназа B\newline
mTORC - Мишень рапамицина млекопитающих\newline
T24 - клеточная линия мочевого пузыря\newline
MDA 468 - клеточная линия карциномы груди\newline
MDA 435 - клеточная линия карциномы груди\newline
SAHA - субероиланилидгидроксамовая кислота\newline
ER - рецептор эстрогена\newline
E2 - 17$\beta$-эстрадиол\newline
LNCaP - клеточная линия рака простаты\newline
PPAR - Рецепторы, активируемые пероксисомными пролифераторами\newline
AD - Болезнь Альцгеймера\newline
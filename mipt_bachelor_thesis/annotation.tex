\documentclass[../main.tex]{subfiles}

\section{Аннотация}

Одним из первых систематических подходов для определения новых применений существующих лекарств - подход Connectivity Map. Он сравнивает сигнатуру запроса, которая представляет собой разницу между двумя интересующими состояниями, с сигнатурами, вызванными различными возмущениями (малыми молекулами). Однако этот подход не учитывает биологическое значение генов в сигнатуре. Мы разработали подход Connectivity Map, основанный на топологических метриках белок-белковых взаимодействий и данных регуляторных сетей. Он учитывает биологическую роль генов и может быть использован для подбора химических веществ по желаемому механизму действия. Наш инструмент находит малые молекулы, которые могут спровоцировать рассматриваемые клеточные изменения. В качестве входных данных используются экспрессионные сигнатуры состояний, которые мы хотим обратить или имитировать. Наша основная гипотеза: если сигнатура из базы данных сильно пересекается с сигнатурой запроса, то это вещество вероятно вызовет аналогичный клеточный ответ. При ранжировании малых молекул на основе косинусного расстояния предполагается, что молекулы с наименьшим расстоянием имеют высокую вероятность вызвать желаемый переход. В рамках нашего подхода была проведена масштабная оптимизация для подбора метрики оценки влиятельности генов. Количество малых молекул с экспериментально подтвержденным механизмом действия, вызывающих клеточные превращения, в верхней части ранжированного списка использовалось в качестве метрики качества. 
